\documentclass[12pt,a4paper]{article}
\usepackage{german}
\usepackage{times}
\usepackage{hyperref}
\usepackage{xspace}
\usepackage{microtype}
%\usepackage{doublespace}
%------------------------------------------------------------------------------
%\setstretch{1.0}
\voffset-5mm
\hoffset-5mm
\textwidth17cm
\textheight24cm
\headsep0mm
\headheight0mm
\oddsidemargin0.3mm
\pagestyle{empty}
\parindent0mm
\parskip1ex
%------------------------------------------------------------------------------
%==============================================================================

\providecommand{\etal}[1]{#1\emph{~et~al.\xspace}}
\renewcommand\refname{References}

\begin{document}



\begin{center}
	Master's Thesis at the Pattern Recognition Lab, FAU Erlangen-Nuremberg \hfill \\[5mm]
																				
	\mbox{}\\
	{\Large Learning Encodings using Deep Neural Networks with Generalized Max Pooling Layers}
			
\end{center}
%Body


State-of-the-art systems for writer recognition use the activation features of deep convolutional neural networks (CNN) to compute local image descriptors~\cite{Christlein2017Unsupervised,Christlein2017Encoding}. Subsequently, a global image descriptor, i.e. a single-vector image-level representation, is computed by encoding the obtained CNN activation features. The standard approach to computing a global descriptor of an image given a set of local descriptors consists of two steps: In the \emph{embedding step} an embedding function maps each local descriptor into a high-dimensional space. During the \emph{aggregation step} an aggregation function, e.g. sum pooling or max pooling, computes a single vector from the embedded feature vectors.  The global image descriptors are used to compute the similarity score between two images. 

However, the aggregated descriptors can suffer from interference of unrelated descriptors that influence the similarity, even if they have low individual similarity~\cite{murray2017interferences}. The issue of interference is exacerbated in the presence of bursty descriptors, that means very similar frequent descriptors that together form a mode in descriptor space and therefore dominate the similarity metric. Murray and Perronnin~\cite{murray2014generalized} propose a generalized form of max pooling to address this issue by reducing the influence of bursty descriptors on the global descriptor.
		
		
In this work, generalized max pooling is incorporated as a layer into a deep neural network architecture, allowing for end-to-end training of the network for writer recognition. 
		
The thesis consists of the following milestones:
\begin{itemize}
	\item Incorporating generalized max pooling~\cite{murray2017interferences,murray2014generalized} into a neural network.
	      	      	      	      	      	      	      		      		      	      	      	      	      	      	      	      	      	      	
	\item Evaluating performance on the ICDAR17	competition dataset on historical	document writer identification~\cite{ICDAR2017WI}.
	      	      	      	      	      	      	      		      		      	      	      	      	      	      	      	      	      	      
	\item Comparison with other pooling techniques.
	      	      	      	      	      	      	      		      		      	      	      	      	      	      	      	      	      	      	
	\item Further experiments regarding learning procedure and network architecure.
	      	      	      	      	      	      	      		      		      	      	      	      	      	      	      	      	      	      
\end{itemize}
		
		
The implementation should be done in Python / C++.\\
		
\begin{tabular}{ll}
	\emph{Supervisors:} & Dr.-Ing.~V.~Christlein,  Prof.~Dr.-Ing.~habil.~A.~Maier 
	\\
	\emph{Student:}     & Lukas Spranger                                            
	\\
	\emph{Start:}       & June 1st, 2018                                            \\
	\emph{End:}         & December 1st, 2018                                        \\
\end{tabular}
\nopagebreak[4]
\small
\bibliographystyle{unsrt}       %TODO change bibliographystyle
\bibliography{proposal}
		
\end{document}
%==============================================================================
